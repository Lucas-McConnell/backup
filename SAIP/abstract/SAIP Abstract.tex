\documentclass{article}
\title{Optimising analysis working points and cuts in same-sign $WW \longrightarrow WW$ scattering}
\begin{document}
\maketitle
\begin{abstract}
$ W^{\pm}W^{\pm} \longrightarrow W^{\pm}W^{\pm}$ is a rare Standard Model process which can be used to investigate the spontaneous symmetry breaking present in the Standard Model. Events with two reconstructed same sign leptons ($e^{\pm}e^{\pm}$,$e^{\pm}\mu^{\pm}$, and $\mu^{\pm}\mu^{\pm}$) and two jets are considered. Background leptons coming from the decay of a jet are said to be non-prompt leptons and make up the jet faked background. Suppression of the non-prompt background is done using a veto on events containing a b-jet and by imposing an isolation requirement on lepton candidates. We optimise the b-jet tagger working point and use the cumulative significance variable in order to determine the optimal working point for the ATLAS isolation selection tool. This approach is then extended to optimising other analysis cuts.
\end{abstract}

\end{document}