\chapter{Additional Comparison Plots}
\label{appendix3}
\begin{figure}
\centering
\begin{subfigure}{.85\textwidth}
  \centering
  \includegraphics[width=1.\linewidth]{../Desktop/workspace/l0-cone20/topoetcone20.pdf}
  \caption{}
  \label{leading_topoetcone}
\end{subfigure}
\begin{subfigure}{.85\textwidth}
  \centering
  \includegraphics[width=1.\linewidth]{../Desktop/workspace/l1-cone20/topoetcone20.pdf}
  \caption{}
  \label{subleading_topoetcone}
\end{subfigure}
\caption{Comparison of the \textbf{topoetcone20} obtained cumulative significances for the leading (a) and sub-leading (b) lepton candidates; obtained for the case of requiring the leptons to pass the \textit{Gradient} isolation requirement (dashed lines) and for the case where the leptons are not required to pass any isolation requirement (solid lines).}
\label{comp_topoetcone}
\end{figure}
\begin{figure}
\centering
\begin{subfigure}{.85\textwidth}
  \centering
  \includegraphics[width=1.\linewidth]{../Desktop/workspace/l0-cone20/frac-topoetcone20.pdf}
  \caption{}
  \label{leading_frac-topoetcone}
\end{subfigure}
\begin{subfigure}{.85\textwidth}
  \centering
  \includegraphics[width=1.\linewidth]{../Desktop/workspace/l1-cone20/frac-topoetcone20.pdf}
  \caption{}
  \label{subleading_frac-topoetcone}
\end{subfigure}
\caption{Comparison of the \textbf{fractional-topoetcone20} obtained cumulative significances for the leading (a) and sub-leading (b) lepton candidates; obtained for the case of requiring the leptons to pass the \textit{Gradient} isolation requirement (dashed lines) and for the case where the leptons are not required to pass any isolation requirement (solid lines).}
\label{comp_frac-topoetcone}
\end{figure}
\begin{figure}
\centering
\begin{subfigure}{.85\textwidth}
  \centering
  \includegraphics[width=1.\linewidth]{../Desktop/workspace/l0-cone20/ptvarcone20.pdf}
  \caption{}
  \label{leading_ptvarcone}
\end{subfigure}
\begin{subfigure}{.85\textwidth}
  \centering
  \includegraphics[width=1.\linewidth]{../Desktop/workspace/l1-cone20/ptvarcone20.pdf}
  \caption{}
  \label{subleading_ptvarcone}
\end{subfigure}
\caption{Comparison of the \textbf{ptvarcone20} obtained cumulative significances for the leading (a) and sub-leading (b) lepton candidates; obtained for the case of requiring the leptons to pass the \textit{Gradient} isolation requirement (dashed lines) and for the case where the leptons are not required to pass any isolation requirement (solid lines).}
\label{comp_ptvarcone}
\end{figure}
\begin{figure}
\centering
\begin{subfigure}{.85\textwidth}
  \centering
  \includegraphics[width=1.\linewidth]{../Desktop/workspace/l0-cone20/frac-ptvarcone20.pdf}
  \caption{}
  \label{leading_frac-ptvarcone}
\end{subfigure}
\begin{subfigure}{.85\textwidth}
  \centering
  \includegraphics[width=1.\linewidth]{../Desktop/workspace/l1-cone20/frac-ptvarcone20.pdf}
  \caption{}
  \label{subleading_frac-ptvarcone}
\end{subfigure}
\caption{Comparison of the \textbf{fractional-ptvarcone20} obtained cumulative significances for the leading (a) and sub-leading (b) lepton candidates; obtained for the case of requiring the leptons to pass the \textit{Gradient} isolation requirement (dashed lines) and for the case where the leptons are not required to pass any isolation requirement (solid lines).}
\label{comp_frac-ptvarcone}
\end{figure}
\begin{figure}
\centering
\begin{subfigure}{.85\textwidth}
  \centering
  \includegraphics[width=1.\linewidth]{../Desktop/workspace/l0-cone40/topoetcone40.pdf}
  \caption{}
  \label{leading_topoetcone}
\end{subfigure}
\begin{subfigure}{.85\textwidth}
  \centering
  \includegraphics[width=1.\linewidth]{../Desktop/workspace/l1-cone40/topoetcone40.pdf}
  \caption{}
  \label{subleading_topoetcone}
\end{subfigure}
\caption{Comparison of the \textbf{topoetcone40} obtained cumulative significances for the leading (a) and sub-leading (b) lepton candidates; obtained for the case of requiring the leptons to pass the \textit{Gradient} isolation requirement (dashed lines) and for the case where the leptons are not required to pass any isolation requirement (solid lines).}
\label{comp_topoetcone}
\end{figure}
\begin{figure}
\centering
\begin{subfigure}{.85\textwidth}
  \centering
  \includegraphics[width=1.\linewidth]{../Desktop/workspace/l0-cone40/frac-topoetcone40.pdf}
  \caption{}
  \label{leading_frac-topoetcone}
\end{subfigure}
\begin{subfigure}{.85\textwidth}
  \centering
  \includegraphics[width=1.\linewidth]{../Desktop/workspace/l1-cone40/frac-topoetcone40.pdf}
  \caption{}
  \label{subleading_frac-topoetcone}
\end{subfigure}
\caption{Comparison of the \textbf{fractional-topoetcone40} obtained cumulative significances for the leading (a) and sub-leading (b) lepton candidates; obtained for the case of requiring the leptons to pass the \textit{Gradient} isolation requirement (dashed lines) and for the case where the leptons are not required to pass any isolation requirement (solid lines).}
\label{comp_frac-topoetcone40}
\end{figure}
\begin{figure}
\centering
\begin{subfigure}{.85\textwidth}
  \centering
  \includegraphics[width=1.\linewidth]{../Desktop/workspace/l0-cone40/ptvarcone40.pdf}
  \caption{}
  \label{leading_ptvarcone}
\end{subfigure}
\begin{subfigure}{.85\textwidth}
  \centering
  \includegraphics[width=1.\linewidth]{../Desktop/workspace/l1-cone40/ptvarcone40.pdf}
  \caption{}
  \label{subleading_ptvarcone}
\end{subfigure}
\caption{Comparison of the \textbf{ptvarcone40} obtained cumulative significances for the leading (a) and sub-leading (b) lepton candidates; obtained for the case of requiring the leptons to pass the \textit{Gradient} isolation requirement (dashed lines) and for the case where the leptons are not required to pass any isolation requirement (solid lines).}
\label{comp_ptvarcone40}
\end{figure}
\begin{figure}
\centering
\begin{subfigure}{.85\textwidth}
  \centering
  \includegraphics[width=1.\linewidth]{../Desktop/workspace/l0-cone40/frac-ptvarcone40.pdf}
  \caption{}
  \label{leading_frac-ptvarcone}
\end{subfigure}
\begin{subfigure}{.85\textwidth}
  \centering
  \includegraphics[width=1.\linewidth]{../Desktop/workspace/l1-cone40/frac-ptvarcone40.pdf}
  \caption{}
  \label{subleading_frac-ptvarcone}
\end{subfigure}
\caption{Comparison of the \textbf{fractional-ptvarcone40} obtained cumulative significances for the leading (a) and sub-leading (b) lepton candidates; obtained for the case of requiring the leptons to pass the \textit{Gradient} isolation requirement (dashed lines) and for the case where the leptons are not required to pass any isolation requirement (solid lines).}
\label{comp_frac-ptvarcone40}
\end{figure}