\chapter{Gauge Theories}
\label{appendix1}
This section is adapted from \cite{griffiths}.

The Lagrangian formulation of classical mechanics has at its core the \emph{Euler-Lagrange} equations:
\begin{equation}
\frac{d}{dt} \left( \frac{\partial \mathcal{L}}{\partial \dot{q_{j}}} \right) = \frac{\partial \mathcal{L}}{\partial q_{j}} \ \ \ \ \ \ \ \ \ \ \ \ (j=1,2,3,...),
\end{equation}
where $ \mathcal{L} = T - U$, $T$ is the kinetic energy of the particle, $U$ is a scalar potential energy, and $q_{j}$ are the generalised coordinates.

In relativistic field theory space and time must be treated equally and the Euler-Lagrange equations generalise to:
\begin{equation}
\partial _{\mu} \left( \frac{\partial \mathcal{L}}{\partial \left( \partial_{\mu} \phi_{i} \right)} \right) = \frac{\partial \mathcal{L}}{\partial \phi_{i}} \ \ \ \ \ \ \ \ \ \ \ \ (i=1,2,3,...),
\end{equation}
where $\phi _{i}$ are the field variables which are now functions of space and time.

The Lagrangian in relativistic field theories is usually taken to be axiomatic, in contrast to how it is usually derived classically. As an example consider the \emph{Dirac Lagrangian}:
\begin{equation}
\mathcal{L} = i \bar{\psi} \gamma _{\mu} \partial_{\mu} \psi - m \bar{\psi} \psi,
\end{equation}
where $\psi$ is a spinor field, $\bar{\psi}$ its adjoint, and $\gamma ^{\mu}$ the gamma matrices. Inserting this into the generalised Euler-Lagrange equations produces the \emph{Dirac equation}, which describes a massive spin-1/2 particle. Note that the Dirac Lagrangian is invariant under what is called a \emph{global phase transformation}: 
\begin{equation}
\psi \longrightarrow e^{i \theta} \psi,
\label{gpt}
\end{equation}
where the phase factor $\theta \in \mathbb{R}$. However if the phase factor in the transformation is given spatial dependence:
\begin{equation}
 \psi \longrightarrow e^{i \theta (x)} \psi,
\end{equation} 
in what is called now called a \emph{local phase transformation}, then the Lagrangian is no longer invariant since:
$$
\partial_{\mu} \left( e^{i \theta (x)} \psi \right) = i \left( \partial_{\mu} \theta(x) \right) e^{i \theta(x)} \psi + e^{i \theta (x)}\partial_{\mu} \psi.
$$
Defining now: 
\begin{equation}
\lambda (x) = - \frac{\theta(x)}{q},
\end{equation}
where q is the charge of the particle in question, then the local phase transformation becomes:
\begin{equation}
\psi \longrightarrow e^{-iq\lambda (x)}\psi,
\end{equation}
and the Lagrangian gains an extra term when subjected to a local phase transformation:
\begin{equation}
\mathcal{L} \longrightarrow \mathcal{L} + \left( q \bar{\psi}\gamma^{\mu} \psi \right) \partial_{\mu}\lambda
\end{equation}
Introduce the vector field $A_{\mu}$ that transforms under a local phase transformation according to:
\begin{equation}
A_{\mu} \longrightarrow A_{\mu} + \partial_{\mu} \lambda.
\label{vf lpt}
\end{equation}
The Dirac Lagrangian can thus be modified to be invariant under a local phase transformation by adding a term:
\begin{equation}
\mathcal{L} = \left[ i \bar{\psi} \gamma^{\mu} \partial_{\mu} \psi - m \bar{\psi} \psi \right] - \left( q \bar{\psi} \gamma^{\mu} \psi \right) A_{\mu}.
\end{equation}
The last term has the interpretation of the vector field $A_{\mu}$ coupling to the field $\psi$. The full Lagrangian should also contain a free term for $A_{\mu}$. Comparing to the Proca Lagrangian\footnote{When entered into the Euler-Lagrange equations, the Proca Lagrangian yields the Proca equation (which describes a massive particle of spin-1).}:
\begin{equation}
\mathcal{L} = \frac{-1}{16 \pi} F^{\mu \nu} F_{\mu \nu} + \frac{1}{8 \pi} \left( m_{A} \right)^{2} A^{\nu} A_{\nu},
\end{equation}
where $m_{A}$ is the mass of the vector field and 
\begin{equation}
F^{\mu \nu} = \left( \partial^{\mu} A^{\nu} - \partial^{\nu} A^{\mu} \right).
\end{equation}
$F{\mu \nu}$ is the field strength tensor and it is invariant under a local phase transformation, however $A^{\nu}A_{\nu}$ isn't unless $m_{A} = 0$. Thus in order for the free term to be invariant under a local phase transformation, the vector field must be massless. If this is interpreted in the context of electrodynamics then $A^{\mu}$ is the electromagnetic potential and equations \ref{gpt} and \ref{vf lpt} are gauge transformations analogous to the ones found in classical electrodynamics. Additionally, in this context, vector fields such as $A^{\mu}$ are commonly referred to as gauge fields.

The local phase invariance can be regained by replacing the derivatives $\partial_{\mu}$ with so-called \emph{covariant derivatives}:
\begin{equation}
D_{\mu} \equiv \partial_{\mu} + iqA_{\mu},
\end{equation}
as this will cancel the extra term introduced in equation \ref{vf lpt} such that:
\begin{equation}
D_{\mu} \psi \longrightarrow e^{iq \lambda} D_{\mu} \psi,
\end{equation}
and the Lagrangian will now be local phase invariant.

The gauge transformations can be thought of as multiplication of the field $\psi$ by a unitary $1 \times 1$ matrix:
\begin{equation}
\psi \longrightarrow U \psi,
\end{equation}
where $ U = e^{i \theta}$. This Quantum Field Theory (QFT) is an example of what is commonly referred to as a \emph{gauge theory}. Gauge theories are a special class of QFTs that are closely associated with a particular symmetry group. In the case here the symmetry group is $\mathbf{U}(1)$. Such symmetry groups can be either abelian (as is the case here) or non-abelian.\footnote{Gauge theories that are associated with non-abelian symmetry groups are commonly referred to as Yang-Mills theories; e.g. in QCD the symmetry group is the non-abelian $SU(3)$, as such QCD is a Yang-Mills theory.}

The field variables in a Standard Model gauge theory are interpreted as being quantised, with the quanta of the fields being particles. Hence, the quantum of the electromagnetic field $A^{\mu}$ is the photon, leptons and quarks are associated with Dirac fields, gluons are the quanta of the eight $\mathbf{SU}(3)$ gauge fields of QCD, and the W and Z bosons are the quanta of associated Proca fields. In particular, the quanta of the gauge fields are the gauge bosons; spin-1 particles which mediate the interaction associated with their field.