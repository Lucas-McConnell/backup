\chapter{Conclusion}
The scattering of W-bosons is a key process in probing electroweak symmetry breaking. In the absence of a Standard Model Higgs boson, the longitudinally polarised scattering amplitude of W-bosons scattering violates unitarity when the WW centre-of-mass energy exceeds approximately 1 TeV. A mechanism is required to unitarise this process. By adding the Standard Model Higgs boson, the cross section regains reasonable behaviour at high energies and the mathematical consistency of the Standard Model is preserved via the Higgs mechanism. A prompt lepton is defined as a lepton coming from a W-boson, while a non-prompt lepton is defined as coming from the decay of a hadron. Processes that contain non-prompt leptons form part of the background in events selected for the same sign W-boson scattering measurement. These non-prompt leptons are called the `fake lepton background'. The dominant contribution to the fake lepton background is from the process:$t\bar{t} \longrightarrow WbWb \longrightarrow \ell \nu  bb qq$.

In this thesis, two strategies for suppressing the fake lepton background were optimised. Optimisation studies on the ATLAS Run 2 multivariate b-tagger suggest moving the working point from the current 85$\%$ efficiency, currently used by the ATLAS Run 2 same-sign W-boson scattering analysis, to the 70$\%$ efficiency in order to maximise the significance. However, due to the fact that the effect on increasing the significance may be a statistical fluctuation, more studies using larger sample sizes should be conducted before recommending the change.  It was also determined that the \textit{Gradient} isolation selection working point was the optimal requirement that can be imposed on signal lepton candidates; no additional fixed cut on the \textbf{topoetcone} or \textbf{ptvarcone} isolation variables is necessary. The isolation requirement was optimised using the cumulative significance quantity, which was first tested in order to anticipate its usefulness for determining cuts on analysis variables.

This approach using the cumulative significance was then applied to three other analysis cuts: the lepton invariant mass $m_{\ell \ell}$, the jet invariant mass $m_{jj}$, and the jet rapidity separation $\Delta y_{jj}$. The advantage of implementing a Z-mass veto on the $ee$-channel was confirmed by examining the cumulative significance of $m_{\ell \ell}$. This examination also indicated that a Z-mass veto should also be implemented on the $\mu\mu$-channel. The studies indicated that jet invariant mass, jet rapidity separation, and $m_{\ell \ell}$ ($e\mu + \mu e$)-channel do not require cuts. These cut optimisations are not in agreement with those utilised for the ATLAS Run 1 same-sign W-boson scattering analysis. It is possible that the Monte Carlo simulated data being used for the optimisation in this thesis is not accurately modeling the same-sign W-boson scattering process or one or more of the background processes. Further analysis using additional MC samples is needed in order to vindicate or dismiss the cut recommendations on the $m_{\ell \ell}$, $m_{jj}$, and  $\Delta y_{jj}$ variable. Additionally, the effects of the recomendations on actual LHC data recorded by the ATLAS detector could be insightful as to whether they are advantageous.