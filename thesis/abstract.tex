\begin{abstract}
Same-sign W-boson scattering is a rare Standard Model process that is useful for probing the nature of electroweak symmetry breaking and the Higgs mechanism. Analysis is currently underway to measure the cross-section to a significance of $5 \sigma$ or higher using $\sqrt{s} = 13$ TeV data from the ATLAS detector's Run 2. The two scattered W-bosons decay leptonically leaving a distinctive experimental signature of two same-sign leptons, two forward jets, and missing transverse energy carried away by two neutrinos. Non-prompt leptons are defined as leptons coming from the decay of hadrons. Such leptons, together with jets misreconstructed as leptons, contribute to the background processes in same-sign W-boson scattering; making up the so-called \emph{fake lepton background}. In this thesis the fake lepton background is suppressed using two strategies: 1) implementing an optimised veto on events found to contain a b-jet; and 2) optimising the isolation requirements set on signal lepton candidates using the cumulative significance quantity. The approach using the cumulative significance is then extended to optimise additional analysis cuts on the lepton invariant mass $m_{\ell \ell}$, jet invariant mass $m_{jj}$, and the jet separation rapidity $\Delta y_{jj}$.
\end{abstract}